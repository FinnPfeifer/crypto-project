\documentclass[conference]{IEEEtran}
\IEEEoverridecommandlockouts
% The preceding line is only needed to identify funding in the first footnote. If that is unneeded, please comment it out.
\usepackage{cite}
\usepackage{amsmath,amssymb,amsfonts}
\usepackage{csquotes}
\usepackage{algorithmic}
\usepackage{graphicx}
\usepackage{textcomp}
\usepackage{xcolor}
\def\BibTeX{{\rm B\kern-.05em{\sc i\kern-.025em b}\kern-.08em
    T\kern-.1667em\lower.7ex\hbox{E}\kern-.125emX}}
\begin{document}

\title{Security of OpenSSL Tool}

\author{\IEEEauthorblockN{John Phillips}
\textit{johphill@mines.edu}\\
\and
\IEEEauthorblockN{2\textsuperscript{nd} Abhaya Shrestha}
\IEEEauthorblockA{\textit{ashrestha@mines.edu}}
\and
\IEEEauthorblockN{3\textsuperscript{rd} Joe Granmoe}
\IEEEauthorblockA{\textit{jgranmoe@mines.edu}}
\and
\IEEEauthorblockN{4\textsuperscript{th} Patrick Curran}
\IEEEauthorblockA{\textit{pcurran@mines.edu}}
\and
\IEEEauthorblockN{5\textsuperscript{th} Collin McDade}
\IEEEauthorblockA{\textit{collinmcdade@mines.edu}}
\and
\IEEEauthorblockN{6\textsuperscript{th} Noor Malik}
\IEEEauthorblockA{\textit{nmalik@mines.edu}}
}

\maketitle

\begin{abstract}
There have been small but catastrophic modifications to openssl in the
past\cite{1}\cite{2}\cite{5} that cause massive security problems and
in some cases exploits that can disclose private keys. We explore in
depth one such instance, a modification made to the OpenSSL package
distributed in debian that resulted in only 32,768 possible keys that
could be generated. We also explore similarities to
heartbleed. Finally, we investigate methods that have been put in
place to prevent such bugs from happening in the future.
\end{abstract}

\begin{IEEEkeywords}
OpenSSL, security vulnerability, debian
\end{IEEEkeywords}

\section{Introduction}
\section{Background}
In 2008 a debian developer made a patch to openssl to fix valgrind
warnings\cite{2}\cite{3}. OpenSSL has a flexible system for generating
random numbers that uses an interface-like approach with functions
prefixed with \verb|RAND_*| delegating to a particular
implementation. Since this API was designed for cryptographic
applications, the random bytes generated need to not just satisfy
particular statistical properties but be truly unpredictable. In order
to acheive this property, this particular random number generator
allows the user to add 'entropy' by feeding it buffers full of random
bytes. These bytes could be from \verb|/dev/urandom|, the user, or
somewhere else. When adding entropy in this way to the message
digest-based random API, an internal hash state is update each time
with the random bytes given by the user. The hash state then can be
used to get random bytes back out of the system. 

Now this particular implementation in \verb|md_rand.c|,
\emph{deliberately} would `over-read' the user-provided buffer. The
apparent thinking behind this was that if the code read uninitialized
memory, this couldn't hurt things, and could only possibly help by
potentially adding randomness.

This behavior occurred in two places in \verb|md_rand.c|. Both of
those places were actually critical to the entire random number
generator. Because of the reading of uninitialized memory, any time a
user of the ssl library did almost anything -- like generate a key,
for example -- they would get disturbing warnings if they ran their
program in the popular \verb|valgrind| tool.

Most linux distributions maintain small set of `patches' for each
package they distribute. These patches normally are designed to fix
critical bugs that haven't been fixed upstream or are in the process
of being fixed. So although the debian project doesn't maintain
OpenSSL, they could patch the particular version of OpenSSL
distributed by debian. And that was the solution arrived at in the bug
ticket and over email \cite{2}\cite{4}.

This was the source of a complaint that first came to the debian bug
tracker\cite{2}. The debian developer quickly spotted the culprit code
but didn't know what it did. Observing that commenting it out didn't
cause anything to crash or obviously function incorrectly, he asked on
the openssl development mailing list if it was okay to remove the two
lines of code -- and got the go-ahead.

\begin{displayquote}
\begin{verbatim}
List:       openssl-dev
Subject:    Re: Random number generator, uninitialised data and valgrind.
From:       Ulf_Möller <ulf () openssl ! org>
Date:       2006-05-01 22:34:12
Message-ID: 44568CE4.9020906 () openssl ! org
[Download RAW message or body]

Kurt Roeckx schrieb:
> What I currently see as best option is to actually comment out
> those 2 lines of code.  But I have no idea what effect this
> really has on the RNG.  The only effect I see is that the pool
> might receive less entropy.  But on the other hand, I'm not even
> sure how much entropy some unitialised data has.
>   
Not much. If it helps with debugging, I'm in favor of removing them. 
(However the last time I checked, valgrind reported thousands of bogus 
error messages. Has that situation gotten better?)
\end{verbatim}
\end{displayquote}
\cite{4}

The debian developer then applied the patch and distributed the debian
version of OpenSSL with this critical vulnerability built in. This
vulnerability was distributed to all users of debian from 2006 to
2008\cite{1}. During that entire time, any code using the random
number generator could only really have 32,767 possible outputs from
the random number generator. Examples of affected functions included
TLS modules and SSH key generation -- actually, anything that
generated RSA keys was vulnerable.

\section{Recreating the 2008 bug}

In order to gain a better understanding of exactly how the bug worked,
and how easy (or difficult) it would be to exploit such a flaw, we
used virtual machines and code from the 2008 era to see what
exploiting the vulnerability might look like. The basic idea was to
create two virtual machines. One would be the exploited machine -- a
stand-in for a server on the Internet or other network that had an
open SSH port that was believed to be secure. The other machine would
try to access the target server over SSH by repeatedly generating
trial keys until it was able to successfully get in. Although this is
not as simple as just writing code to guess the key and compares two
files, it is arguably a more realistic demonstration of how exploits
really work, and a better way of understanding them, putting the
`applied' in `applied cryptography'.

\subsection{Code modifications}
First, we created two different versions of the OpenSSL code. The
first one had the two lines that the debian developer asked about on
the OpenSSL mailling list commented out (see \cite{4}), and nothing
more. It was a re-creation of what existed in the debian project from
2006-2008. We looked for the exact code (for example, a .deb binary
distribution, or the actual patchsets applied to the OpenSSL project)
on the Internet but were unable to find it (links pointing to the
patches were all dead, possibly in a deliberate attempt by debian to prevent
anyone accidentally using the code again).

The next branch of the OpenSSL code was similar to the first except it
had a way to manually set the only real source of variablility left to
the random number generator -- the PID (process ID). We did this by
replacing calls to \verb|getpid()| in \verb|md_rand.c| to references
to a static variable in \verb|md_rand.c| called \verb|fake_pid|. We
also introduced a globally-visible function called
\verb|RAND_exploit_set_pid()|, to allow client code to set the PID.


\subsection{Virtual machine setup}
We built two virtual machines based on debian 5. The machines were
connected to a single virtual switch, and had static IPs assigned. On
one virtual machine, we built and installed the re-created
debian-distributed OpenSSL. We used \verb|ssh-keygen| to create and
authorize a key as an authorized SSH key on that host.

On the other (attacker) machine, we used a program derived from
\verb|ssh-keygen| to try to SSH to the target machine with a given PID
seed. That program then was repeatedly invoked by another script that
iterated over all possible PIDs, from 1 to 32,768 (in 2008 that was
the number of possible PIDs).

The results were as expected -- the attacker machine was able to
generate the public/private key pair of the target machine within
fifteen to twenty minutes. On a fresh boot, because there haven't been
as many processes run, it takes only about five minutes.

\subsection{Running the exploit}

\subsection{Alternative approaches}

\section{Similarities to Heartbleed}
The debian bug bears certain similarities to another bug that occurred
in 2014 -- Heartbleed\cite{4}. Both bugs compromised private keys, and
both were not the result of any protocol design errors, cryptanalytic
breakthroughs, or any other issue commonly learned about in a
cryptography class. Instead, both errors were the result of human
error. Simple, and (in hindsight) rather obvious bugs made their way
into OpenSSL packages that were distributed to <how many>
machines<CITE!>.

Both bugs were the result of reading into memory regions that were not
intended to be read into, which wouldn't even be possible in many
other languages.

This raises the question of how, if possible, to catch such simple
implementation errors?

\section{Investigating the current code}

\section{Processes in the OpenSSL project in place to prevent further exploits}

\begin{thebibliography}{00}
\bibitem{1} \verb|https://isotoma.com/blog/2008/05/14/debians-openssl-disaster/|
\bibitem{2} \verb|https://bugs.debian.org/cgi-bin/bugreport.cgi?bug=363516|
\bibitem{3} \verb|https://research.swtch.com/openssl|
\bibitem{4} \verb|https://marc.info/?l=openssl-dev&m=114652287210110&w=2|
\bibitem{5}  \verb|https://heartbleed.com/|
\end{thebibliography}

\end{document}
