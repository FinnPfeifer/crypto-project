\documentclass[conference]{IEEEtran}
\IEEEoverridecommandlockouts
% The preceding line is only needed to identify funding in the first footnote. If that is unneeded, please comment it out.
\usepackage{cite}
\usepackage{amsmath,amssymb,amsfonts}
\usepackage{algorithmic}
\usepackage{graphicx}
\usepackage{textcomp}
\usepackage{xcolor}
\def\BibTeX{{\rm B\kern-.05em{\sc i\kern-.025em b}\kern-.08em
    T\kern-.1667em\lower.7ex\hbox{E}\kern-.125emX}}
\begin{document}

\title{Security of OpenSSL Tool}

\author{\IEEEauthorblockN{John Phillips}
\textit{johphill@mines.edu}\\
\and
\IEEEauthorblockN{2\textsuperscript{nd} Abhaya Shrestha}
\IEEEauthorblockA{\textit{ashrestha@mines.edu}}
\and
\IEEEauthorblockN{3\textsuperscript{rd} Joe Granmoe}
\IEEEauthorblockA{\textit{jgranmoe@mines.edu}}
\and
\IEEEauthorblockN{4\textsuperscript{th} Patrick Curran}
\IEEEauthorblockA{\textit{pcurran@mines.edu}}
\and
\IEEEauthorblockN{5\textsuperscript{th} Collin McDade}
\IEEEauthorblockA{\textit{collinmcdade@mines.edu}}
\and
\IEEEauthorblockN{6\textsuperscript{th} Noor Malik}
\IEEEauthorblockA{\textit{nmalik@mines.edu}}
}

\maketitle

\begin{abstract}
There have been small but catastrophic modifications to openssl in the
past\cite{1}\cite{2} that drastically reduced the potential number of
keys that could be generated during key generation in the debian
distribution. We'd like to explore one such instance, and see if a
similar vulnerability that is easy to exploit could still be
introduced into the codebase with a small modification to the base
source of OpenSSL.
\end{abstract}

\begin{IEEEkeywords}
OpenSSL, security vulnerability, debian
\end{IEEEkeywords}

\section{Introduction}
\subsection{Background}
In 2008 a debian developer made a patch to openssl to fix valgrind
warnings\cite{2}\cite{3}. Parts of openssl were deliberately using
uninitialized memory to add entropy for key generation in a
nonessential way, and this caused client programs that linked to
openssl to generate error/warnings when developers tried to validate
programs with valgrind. But there was also another function call that
looked very similar elsewhere in the code which was critical to key
generation. The debian developer that worked on the bug (as seen in
\cite{2}) mistakenly thought that \emph{both} function calls should be
removed when only one should have. With both gone, the function
\verb|ssleay_rand_bytes|, which was used to return random bytes to
client code, suddenly only relied on the current process PID to return
randomness. This meant that now there were only $2^{15}$ possible
values returned into the random buffer, so that programs like
\verb|ssh-keygen| that used that function to generate random bytes for
keys, were suddenly extremely weak.

\subsection{Motivation}
We'd like to dig into the current and past source code of openssl,
understand the first exploit referenced in \cite{1} in the old code,
and see if a similar or identical exploit is still easy to introduce
and reproduce. We'd also like to get a sense of how hard it would be
for a malicious actor to introduce another similar exploit into the
code -- for a similar mistake to happen in 2022. This would involve
researching the past and current workflows for making potentially
security-critical patches to both debian and the OpenSSL project at a
minimum, and potentially other popular linux distributions as well.

Open questions are, for example,

\begin{itemize}
\item What does it take to be a developer on one of these projects?
\item How are patches reviewed?
\item How formal is the patch review process, and have potential bugs
  already been successfully caught and removed before introduction to
  distributed code in the past?
\end{itemize}

\subsection{Goal}
We'd like to reproduce the 2008 bug from the git commit of the version
referenced in \cite{2}. Then, we'd like to see if we can make a
similar modification to the current code and demonstrate a similar
working exploit (i.e. we'd like to derive a key and decrypt a file for
example), and show that it could be very simple for a disgruntled
employee or malicious actor to introduce a small patch to OpenSSL in a
downstream distribution (or potentially to the project itself) that
could make it very vulnerable, or alternatively demonstrate that the
codebase has improved and is much harder to accidentally exploit this
way. We will also investigate the current process in the debian
distribution and the OpenSSL project for adding patches, and report on
what we find.

\section{Idea and Approach}

First, we'd like to compile and demonstrate the vulnerability as it
existed in 2008. Most likely this would involve creating a VM image
with an old debian version, compiling an OpenSSL version with the
patch specified in the debian bug report\cite{3}, linking
\verb|ssh-keygen| to the library (which will be a re-creation of what
was distributed in 2008), and write a simple program to guess the
generated key, since there are only $2^{15}$ possibilities.

Then, we'd like to fork a current openssl version, and see if we can
make a small modification to the code that reduces the possible number
of keys in a way that recreates the 2008 vulnerability. This
modification should be similar to the exploit in \cite{1} in that it
is a small number of lines of code that decreases entropy for key
generation, making it easier to guess the private key generated by
openssl.

Then we would create another small program that can guess the key
generated by the new hacked version, also in a reasonable amount of
time since it knows that the keyspace is now much smaller. We could
then for example forge a signature or decrypt a file encrypted with
the public key.

Next, we will investigate their upstream process on how they
detect vulnerability on their code (do they have an automated,
rigorous code review process etc.). This will show us that despite
the possbility of developer error, how likely are we to see
the vulnerability exposed to production environments.

\section{Expected results}

First, we'd have a runnable 'exploit' that works on the old
debian-modified openssl. This should demonstrate that it only takes a
small change to openssl that could be made in any number of places to
decrease the potential keyspace for private keys.

We also would either demonstrate that that kind of vulnerability is
still easy to introduce, potentially accidentally, or that the
codebase and processes around introducing fixes have been fixed in
debian and/or the openssl project.

As far as the upstream process is concerned, if we find
their process is rigorous, then OpenSSL is fairly secure
for its most recent version (with the given computing power
at the moment of the experiment). Otherwise, we will recommend
some other alternaitve to OpenSSL.

\section{Conclusion}
We'd like to create an easy-to-exploit vulnerability in the current
openssl source that is similar to the one inadvertently introduced
into the debian source code in 2008 as documented in\cite{1}. If it's still possible to
make such a change, this raises concerns about the security of an open
source security-critical system like openssl.
If we are not able to find easy-to-exploit vulnerability
and our inverstigation of their contribution process
shows rigorous code reviews and automation in place, 
we will conclude that current
version of OpenSSL (with the given computing power) is 
a secure library for encryption. Otherwise, we will
recommend other libraries to use as an alternative
to openssl.

\begin{thebibliography}{00}

\bibitem{1} https://isotoma.com/blog/2008/05/14/debians-openssl-disaster/
\bibitem{2} https://bugs.debian.org/cgi-bin/bugreport.cgi?bug=363516
\bibitem{3} https://research.swtch.com/openssl  
\end{thebibliography}

\end{document}
